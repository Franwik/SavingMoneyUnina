\chapter{Progettazione Concettuale}

\section{Diagramma Delle Classi UML}

\begin{figure}[ht!]
    \centering
    \includegraphics[scale=0.55]{pdfs/UMLdiagram.drawio.pdf}
    \caption{Diagramma UML}\label{UML}
\end{figure}

\section{Diagramma ER (Entità Relazione)}

\begin{figure}[ht!]
    \centering
    \includegraphics[scale=0.7]{pdfs/ERdiagram.drawio.pdf}
    \caption{Diagramma ER}\label{ER}
\end{figure}

\newpage

\section{Ristrutturazione}

\subsection{Attributi multipli}

Per quanto riguarda la gestione di attributi multipli,
abbiamo deciso di gestire l'attributo \textit{category} della tabella
\textbf{Transaction}, originariamente definito come enumerazione,
trasformandolo in una stringa, poiché non abbiamo bisogno di valori
specifici, trattandosi di una categoria personalizzabile.

Invece, per l'attributo \textit{cardType} della tabella \textbf{Card},
è stato deciso di non applicare lo stesso metodo, poiché le tipologie
di carte sono ben definite e non possono essere modificate.

\subsection{Generalizzazioni}

Per la generalizzazione, essendo di tipologia totale e disgiunta,
abbiamo optato per il metodo di eliminare la classe generale.
Abbiamo trasferito tutti gli attributi di essa nelle
classi specializzate, conservando le relative relazioni.

\subsection{Analisi degli identificativi}

Per la maggior parte delle classi, saranno utilizzati come identificativi
attributi già presenti di natura nelle classi stesse, poiché risultano
sufficienti e non richiedono l'uso di una chiave surrogata.
Tuttavia, in alcune classi, sono presenti chiavi surrogate,
identificate con il prefisso \textbf{ID\_}.

\newpage
\subsection{Diagramma UML ristrutturato}

\begin{figure}[ht!]
    \centering
    \includegraphics[scale=0.7]{pdfs/RestructuredUMLdiagram.drawio.pdf}
    \caption{Diagramma UML Ristrutturato}\label{ResUML}
\end{figure}

\newpage
\section{Dizionari}

\subsection{Dizionario delle classi}

\subsection{Dizionario delle associazione}

\subsection{Dizionario dei vincoli}